\section{Conclusion}
We have introduced the \emph{context-free} network (CFN), a CNN whose features can be easily transferred between detection and classification and Jigsaw puzzle reassembly task. The network is trained in an unsupervised manner by using the Jigsaw puzzle as a \emph{pretext} task. The learned features are then used for the classification task of Food recognition competition yielded by Kaggle. We have seen that the training of the Jigsaw task is very difficult and that the implementation of non-trivial models is a bit tricky with the TensorFlow framework. The results showed in \ref{s:experiments} ar far from the one that the authors of the original paper were able to obtain, mainly as regards the target task. Since little time has been given to the training of the fine tuning, the bad performance may not come unexpected, however we believe that the poor performance of the Jigsaw puzzle solver is the mean reason why also the target task performed so badly.